\documentclass{article}
\usepackage{amsmath}
\usepackage{amssymb}
\usepackage{graphicx} % 如果需要插入图示

\begin{document}

\section{序列化黄金噪声生成模型}

\subsection{动机与直觉}
传统的文本到图像扩散模型通常从一个随机高斯噪声 $x_T \sim \mathcal{N}(0, I)$ 开始,逐步去噪以生成图像。论文 \cite{zhou2024golden} 提出,可以使用一个预训练的网络 (NPNet) 将 $x_T$ 映射到一个“黄金噪声” $x'_T$,该噪声包含了与文本提示 $c$ 相关的语义信息,作为扩散过程的新起点,可能提升生成质量。其数据集通过一步去噪和一步加噪(DDIM + DDIM-Inverse)并结合 AI 反馈(如 HPSv2)构建 $(x_T, c, x'_T)$ 数据对。

我们观察到,原始论文中生成 $x'_T$ 的过程(一步 DDIM Denoise + 一步 DDIM Inversion)可以被迭代多次。直觉上,随着迭代次数 $k$ 的增加,得到的噪声 $x'_k$ 会逐步融入更多与文本提示 $c$ 相关的结构和语义信息,形成一个从随机噪声向“理想”条件化噪声演化的序列 $[x'_1, x'_2, \dots, x'_n]$。这个演化过程可以被建模为一个受文本条件 $c$ 引导的一阶马尔可夫链,即当前噪声 $x'_k$ 的生成主要依赖于上一步的噪声 $x'_{k-1}$ 和文本条件 $c$:
\begin{equation}
    p(x'_k | x'_{k-1}, x'_{k-2}, \dots, x'_1, x_T, c) \approx p(x'_k | x'_{k-1}, c)
\end{equation}
这种序列依赖和条件生成的特性非常适合使用循环神经网络 (RNN) 或其变体(如 GRU, LSTM)进行建模。

此外,为了确保生成的序列噪声 $x'_k$ 能够作为扩散模型的有效起点,它们应保持接近标准高斯分布 $\mathcal{N}(0, I)$ 的统计特性。这需要在模型设计和训练目标中显式地加入约束。

\subsection{模型架构: NoiseSequenceRNN-v3}
基于上述考虑,我们设计了一个基于 GRU 的序列模型 (NoiseSequenceRNN-v3),其目标是学习条件转移概率 $p_\theta(x'_k | x'_{k-1}, c)$。模型结构如下:

\begin{enumerate}
    \item \textbf{噪声编码器 (Noise Encoder):} 使用一个基于 ResNet 风格块(包含组归一化 GroupNorm 和 SiLU 激活函数)的卷积神经网络 (CNN) $E_\phi$ 将上一步的噪声 $x'_{k-1} \in \mathbb{R}^{C \times H \times W}$ 编码为一个低维特征向量 $f_{k-1} \in \mathbb{R}^{D_{feat}}$:
        \begin{equation}
            f_{k-1} = E_\phi(x'_{k-1})
        \end{equation}

    \item \textbf{文本嵌入处理:} 将预训练的文本编码器(如 CLIP)输出的文本嵌入 $c \in \mathbb{R}^{D_{text}}$ (通常使用其 pooled output)通过一个线性层投影到 $c_{proj} \in \mathbb{R}^{D_{proj}}$ (可选,用于维度匹配)。

    \item \textbf{GRU 状态更新:} 将编码后的噪声特征 $f_{k-1}$ 和投影后的文本嵌入 $c_{proj}$ 拼接起来,作为 GRU 单元的输入,更新隐藏状态 $h_k \in \mathbb{R}^{D_{hidden}}$:
        \begin{equation}
            h_k = \text{GRU}( [f_{k-1}; c_{proj}], h_{k-1} )
        \end{equation}
        其中 $[;]$ 表示向量拼接。这使得 RNN 的状态转换同时依赖于前一步的噪声和全局文本条件。

    \item \textbf{FiLM 参数生成:} 使用文本嵌入 $c$ 通过一个或多个线性层(FiLM Generator, $G_\psi$)预测用于特征调制(FiLM)的缩放参数 $\gamma_k$ 和偏移参数 $\beta_k$。这些参数将用于解码器:
        \begin{equation}
            (\gamma_k, \beta_k) = G_\psi(c)
        \end{equation}

    \item \textbf{噪声解码器 (Noise Decoder):} 使用一个与编码器结构对称的 ResNet 风格 CNN 解码器 $D_\omega$,以 GRU 的隐藏状态 $h_k$ 作为输入。在解码器的中间层,应用 FiLM 层,使用步骤 4 生成的 $(\gamma_k, \beta_k)$ 对特征图进行调制。解码器最终输出下一步噪声 $x'_k$ 的分布参数,即均值 $\mu_k \in \mathbb{R}^{C \times H \times W}$ 和对数方差 $\log \sigma^2_k \in \mathbb{R}^{C \times H \times W}$:
        \begin{equation}
            (\mu_k, \log \sigma^2_k) = D_\omega(h_k, \text{FiLM params}=(\gamma_k, \beta_k))
        \end{equation}
        模型直接预测完整状态 $x'_k$ 的分布,而不是残差。

\end{enumerate}

\subsection{训练目标}
模型使用教师强制 (Teacher Forcing) 进行训练。对于序列中的每一步 $k=1, \dots, n$,我们计算两个损失项:

\begin{enumerate}
    \item \textbf{负对数似然损失 (NLL Loss):} 最大化观测到目标噪声 $x'_k$ 的概率,等价于最小化 NLL 损失。假设预测分布为高斯 $\mathcal{N}(\mu_k, \sigma^2_k)$,其中 $\sigma^2_k = \exp(\log \sigma^2_k)$:
        \begin{equation}
            \mathcal{L}_{NLL}^{(k)} = -\log p_\theta(x'_k | x'_{k-1}, c) \propto \frac{1}{2} \sum_{i} \left( \frac{(x'_{k,i} - \mu_{k,i})^2}{\sigma^2_{k,i}} + \log \sigma^2_{k,i} \right)
        \end{equation}
        其中 $i$ 遍历噪声张量的所有元素。

    \item \textbf{KL 散度正则化损失 (KL Loss):} 为了约束预测的噪声分布接近标准高斯分布 $\mathcal{N}(0, I)$,我们计算两者之间的 KL 散度:
        \begin{equation}
            \mathcal{L}_{KL}^{(k)} = D_{KL}(\mathcal{N}(\mu_k, \sigma^2_k) || \mathcal{N}(0, I)) = \frac{1}{2} \sum_{i} (\sigma^2_{k,i} + \mu_{k,i}^2 - 1 - \log \sigma^2_{k,i})
        \end{equation}
\end{enumerate}

总损失是这两项的加权和,对序列中的所有步骤求平均(或加权平均):
\begin{equation}
    \mathcal{L}_{Total} = \frac{1}{n} \sum_{k=1}^{n} (\mathcal{L}_{NLL}^{(k)} + \lambda_{KL} \mathcal{L}_{KL}^{(k)})
\end{equation}
其中 $\lambda_{KL}$ 是 KL 散度损失的权重超参数。

\subsection{推理}
在推理阶段,模型以自回归方式生成噪声序列。从初始噪声 $x_T$ 开始,迭代 $n$ 步:
\begin{enumerate}
    \item 使用当前噪声 $x'_{k-1}$ 和文本嵌入 $c$ 输入模型,得到预测的分布参数 $(\mu_k, \log \sigma^2_k)$。
    \item 从预测的高斯分布 $\mathcal{N}(\mu_k, \exp(\log \sigma^2_k))$ 中采样得到下一步噪声 $\hat{x}'_k$。
    \item 将 $\hat{x}'_k$ 作为下一步的输入。
\end{enumerate}
最终生成的序列为 $[\hat{x}'_1, \dots, \hat{x}'_n]$。通常选择最后一步 $\hat{x}'_n$ 作为输入到基础扩散模型中的“黄金噪声”。

\end{document}
